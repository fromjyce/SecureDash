\documentclass{beamer}
\usetheme{CambridgeUS}
\usepackage{graphicx} % Required for inserting images
\graphicspath{./images}
\usecolortheme{dolphin}

% Define team members
\newcommand{\teamMember}[1]{\textbf{#1}}
\title[IDS in Smart Grid]{Advancing Smart Grid Security with an Intrusion Detection Framework}
\author[Jayashre]{\teamMember{Jayashre}}
\date{\today}


\begin{document}

\begin{frame}
\titlepage
\end{frame}



\begin{frame}
\frametitle{Overview}
\begin{block}{To enhance Smart Grid security, this project develops an advanced intrusion detection system using machine learning to combat cyber threats like DDoS attacks. Its goal is to advance intrusion detection in smart energy grids, ensuring uninterrupted operation and safety in our increasingly digital world.}
\end{block}
\end{frame}

\begin{frame}
\frametitle{Target Audience}
\begin{block}{Utility Companies}
\end{block}
\begin{block}{Cybersecurity Professionals}
\end{block}
\begin{block}{Government Agencies}
\end{block}
\begin{block}{Technology Providers}
\end{block}
\begin{block}{Research \& Academia}
\end{block}
\end{frame}

\begin{frame}
\frametitle{Features}
\begin{block}{Creation of a real-time monitoring dashboard to track network activity and detect anomalies promptly.}
\end{block}
\begin{block}{Integration of the developed system with existing intrusion detection systems for enhanced threat detection and mitigation.}
\end{block}
\begin{block}{Implementation of continuous surveillance techniques to monitor network traffic and identify potential threats in real-time.}
\end{block}
\begin{block}{ Creation of a user-friendly interface for system management and monitoring, facilitating ease of use for grid operators and cybersecurity personnel}
\end{block}
\end{frame}

\begin{frame}
\frametitle{Tech Stack}
\begin{block} {
\begin{itemize}
    \item Python: Used for implementing machine learning and deep learning approaches.
    \item Datasets: Utilizing the CIC-DDOS2019 dataset.
    \item Real-Time Monitoring Platforms: Such as Kafka and Pulsar.
    \item Intrusion Detection and Prevention Systems: Like Suricata and Snort.
    \item Monitoring and Visualization Tools: Including Grafana, Tableau, and Power BI.
    \item Google Colab: For the development of machine learning models in Python.
\end{itemize}}
\end{block}

\end{frame}

\begin{frame}
\frametitle{Future Scope}
\begin{block}
    {Advancing explainable AI techniques is crucial as deep-learning models gain prevalence in critical systems. Ensuring transparent decision-making enhances practicality in real-world scenarios by unraveling the detection process and providing meaningful insights.}
\end{block}
\begin{block}
    {Expanding on this study's foundation to delve deeper into the nuances of intrusion detection within SCADA and Smart Grid settings. }
\end{block}
\end{frame}

%\begin{frame}
%\frametitle{Timeline}

%\%end{frame}


%\begin{frame}
%\frametitle{Project Limitations}
%\begin{block} {Challenges and Progress in Integrating Models with Camera App Development}
%\end{block}
%\end{frame}

%\begin{frame}
%\frametitle{Tech Stack}
%\begin{block}{AR Filters and Colour Filters}
%MediaPipe - Face detection\\
%OpenCV's Facemark API - Face features detection \\
%Makesense - Labelling the feature points in the photo \\
%Photoshop, free to use images - Filters Deploying\\
%Lookup Table Library - Colour Filters\\
%\end{block}
%\end{frame}



%%
%%\begin{frame}
%%\frametitle{Tech Stack}
%%\begin{block}{Integration}
%%Pyjnius - Python - to - Java Bridge\\
%%Setting up Python Environment\\
%%Preparing Python-to-Kotlin integration.\\
%%Python code using Bee Ware Library for Android execution.\\
%%\end{block}
%%\end{frame}

%\begin{frame}
%\frametitle{Demo}
%\begin{block}{We will be demonstrating the following models}
%\begin{itemize}
%    \item AirPic: The App
%    \item Smile Detection Model
%    \item Palm Detection Model
%    \item Gesture Model for Zooming In and Out
%\end{itemize}
%\end{block}
%\end{frame}

%\begin{frame}
%\frametitle{Learnings}
%\end{frame}

%\begin{frame}
%\frametitle{Future Scope}
%\begin{block} {1. To implement customized in-app settings such as grids and HDR.}
%\end{block}
%\begin{block} {2. Create a centralized in-app Gallery for efficient management and viewing of images and videos, enhancing user experience}
%\end{block}
%\begin{block}{3. Develop our project into an iOS App}
%\end{block}
%\begin{block}{4. Enrich user experience by incorporating intuitive gestures like turning on the video or activating the timer.}
%\end{block}
%\begin{block}{5. Transform our project into a vibrant social platform, fostering connections and collaboration.}
%\end{block}
%\end{frame}

\begin{frame}
\frametitle{Conclusion \& Thank You}
\begin{block}
    {I value and appreciate your feedback.}
\end{block}
\end{frame}

\end{document}

